\documentclass{article}

\usepackage[bookmarksnumbered, colorlinks, plainpages]{hyperref}
\usepackage{fancyhdr}
\usepackage{extramarks}
\usepackage{amsmath}
\usepackage{amsthm}
\usepackage{amsfonts}
\usepackage{tikz}
\usepackage[plain]{algorithm}
\usepackage{algpseudocode}
\usepackage{graphicx}

\graphicspath{ {./img/} }
\usetikzlibrary{automata, positioning}

%
% Basic Document Settings
%

\topmargin=-0.45in
\evensidemargin=0in
\oddsidemargin=0in
\textwidth=6.5in
\textheight=9.0in
\headsep=0.25in

\linespread{1.1}

\pagestyle{fancy}
\lhead{\hmwkAuthorName}
\chead{\hmwkClass\ (\hmwkClassInstructor\ \hmwkClassTime): \hmwkTitle}
\rhead{\firstxmark}
\lfoot{\lastxmark}
\cfoot{\thepage}

\renewcommand\headrulewidth{0.4pt}
\renewcommand\footrulewidth{0.4pt}

\setlength\parindent{0pt}

%
% Create Problem Sections
%

\newcommand{\enterProblemHeader}[1]{
    \nobreak\extramarks{}{}\nobreak{}
    \nobreak\extramarks{}{}\nobreak{}
}

\newcommand{\exitProblemHeader}[1]{
    \nobreak\extramarks{}{}\nobreak{}
    \stepcounter{#1}
    \nobreak\extramarks{Problem \arabic{#1}}{}\nobreak{}
}

\setcounter{secnumdepth}{0}
\newcounter{partCounter}
\newcounter{homeworkProblemCounter}
\setcounter{homeworkProblemCounter}{1}
\nobreak\extramarks{Problem \arabic{homeworkProblemCounter}}{}\nobreak{}

%
% Homework Problem Environment
%

\newenvironment{homeworkProblem}[1]{
    %\section{Problem \arabic{homeworkProblemCounter}{#1}}
    \setcounter{partCounter}{1}
    \enterProblemHeader{homeworkProblemCounter}
}{
    \exitProblemHeader{homeworkProblemCounter}
}

%
% Homework Details
%   - Title
%   - Due date
%   - Class
%   - Section/Time
%   - Instructor
%   - Author
%

\newcommand{\hmwkTitle}{Milestone 3}
\newcommand{\hmwkDueDate}{March 5, 2024}
\newcommand{\hmwkClass}{CS 362}
\newcommand{\hmwkClassTime}{11:00am}
\newcommand{\hmwkClassInstructor}{Professor Troy}
\newcommand{\hmwkAuthorName}{\textbf{John Ezra See}\\\textbf{Ryan Magdaleno}}
\newcommand{\hwline}{\begin{center}\line(1,0){358px}\end{center}}

%
% Title Page
%

\title{
    \vspace{2in}
    \textmd{\textbf{\hmwkClass:\ \hmwkTitle}}\\
    \normalsize\vspace{0.1in}\small{Due\ on\ \hmwkDueDate\ at 11:59pm}\\
    \vspace{0.1in}\large{\textit{\hmwkClassInstructor\ \hmwkClassTime}}
    \vspace{3in}
}

\author{\hmwkAuthorName\\
\href{mailto:jsee4@uic.edu}{jsee4@uic.edu}\\
\href{mailto:rmagd2@uic.edu}{rmagd2@uic.edu}}
\date{}

\renewcommand{\part}[1]{\textbf{\large Part \Alph{partCounter}}
\stepcounter{partCounter}\\}
%
% Various Helper Commands
%

% Useful for algorithms
\newcommand{\alg}[1]{\textsc{\bfseries \footnotesize #1}}

% For derivatives
\newcommand{\deriv}[1]{\frac{\mathrm{d}}{\mathrm{d}x} (#1)}

% For partial derivatives
\newcommand{\pderiv}[2]{\frac{\partial}{\partial #1} (#2)}

% Integral ds
\newcommand{\dx}{\mathrm{d}x}
\newcommand{\D}[1]{\mathrm{d}#1}

% Image insertion
\newcommand{\img}[2]{\begin{center}\includegraphics[scale=#1]{#2}\end{center}}

% Alias for the Solution section header
\newcommand{\solution}{\textbf{\large Solution}\\}

% Probability commands: Expectation, Variance, Covariance, Bias
\newcommand{\E}{\mathrm{E}}
\newcommand{\Var}{\mathrm{Var}}
\newcommand{\Cov}{\mathrm{Cov}}
\newcommand{\Bias}{\mathrm{Bias}}

\begin{document}

\maketitle

\pagebreak

%%%%%%%%%%%%%%%%%%%%%%%%%%%%%%%%%%%%%%%%%%%%%%%%%%%%%%%%%%%%%%%%%%%%%%%%%%%%%%%%%%%%%%%%%

\begin{homeworkProblem}{}
    \subsection{Group Information}
    Group 62: VSRG-UNO-R3 \\
    Names:
    \begin{enumerate}
        \item \vspace{-10pt}
        Ryan Magdaleno: \href{mailto:rmagd2@uic.edu}{rmagd2@uic.edu}
        \item \vspace{-10pt}
        John Ezra See: \hspace{11pt}\href{mailto:jsee4@uic.edu}{jsee4@uic.edu}
    \end{enumerate}
    \hwline
    \vspace{-20pt}\subsection{Abstract}
    This project introduces an original Arduino/C++ endeavor aimed at enhancing the 
    gaming experience through innovative hardware and software integration. Utilizing 
    Arduino microcontrollers and solenoid arrays, it offers physical interaction with 
    rhythm games, synchronizing gameplay with real-time data processing. The system 
    manages communication between components and supports customization preferences. 
    Through this project, users can enjoy an immersive and customizable VSRG experience, 
    demonstrating creativity and innovation in Arduino/C++ technology.
    \hwline
\end{homeworkProblem}

\pagebreak

%%%%%%%%%%%%%%%%%%%%%%%%%%%%%%%%%%%%%%%%%%%%%%%%%%%%%%%%%%%%%%%%%%%%%%%%%%%%%%%%%%%%%%%%%

\begin{homeworkProblem}{}
    \subsection{1 Project Idea}
    \hspace{20pt}Our project idea centers around enhancing the experience of playing 
    Vertical Scrolling Rhythm Games (VSRGs), particularly focusing on key presses. We 
    plan to utilize two Arduino microcontrollers to manage two displays and a solenoid 
    array. One Arduino will control the displays, showcasing both the keys pressed per 
    second and the key press strokes. This data will be calculated and sent via serial 
    communication to a computer-side program. We intend to update the display with the 
    keys per second value every 20 milliseconds on a 16x2 LCD screen. Additionally,
    we will transmit a binary value representing the keys pressed, also updating every 20 
    milliseconds. 

    \vspace{10pt}\hspace{20pt}For gameplay, we've chosen Osu Mania due to its easily 
    parseable .osu file format, which contains hit timings for beatmaps. Although Osu 
    Mania supports maps of various key counts, we will focus on maps ranging from 4 to 7 
    keys. On the computer side, the program will read the hit timings from the .osu files 
    and organize them into a hit-chart container. It will then gradually transmit this 
    data to the solenoid array Arduino for physical key presses. We're also considering 
    utilizing the Arduino's flash memory to store this data, although this aspect is 
    still under consideration. The solenoid array Arduino will receive the .osu data and 
    trigger the solenoids to execute physical key presses. The computer program will 
    continuously monitor these key presses, calculating both keystrokes and keys per 
    second, and sending the accurate values to the display Arduino. For the Arduino 
    managing keystrokes, we will employ a Nokia 5110 LCD to display the pressed keys,
    animating them upwards for visual feedback.
    \hwline
\end{homeworkProblem}


%%%%%%%%%%%%%%%%%%%%%%%%%%%%%%%%%%%%%%%%%%%%%%%%%%%%%%%%%%%%%%%%%%%%%%%%%%%%%%%%%%%%%%%%%

\begin{homeworkProblem}{}
    \vspace{-20pt}\subsection{2 Initial Project Design}
    \hspace{20pt}One Arduino will receive hit press data from a computer-side program and 
    activate corresponding solenoid keys. The other Arduino will receive keyboard press 
    information, including keys per second and recent key presses. The computer-side 
    program will start both Arduinos upon pressing key H and stop them upon pressing key 
    J. A Serial connection will be transmitted from the solenoid Arduino to the display 
    Arduino, terminating both Arduinos simultaneously.
    \hwline
\end{homeworkProblem}

%%%%%%%%%%%%%%%%%%%%%%%%%%%%%%%%%%%%%%%%%%%%%%%%%%%%%%%%%%%%%%%%%%%%%%%%%%%%%%%%%%%%%%%%%

\begin{homeworkProblem}{}
    \vspace{-20pt}\subsection{3 Arduino Communication}
    \hspace{20pt}For communication, we'll utilize a computer-side program to send the 
    necessary data to each Arduino as required. The solenoid Arduino will receive beatmap 
    hit data, while the display Arduino will receive keys per second (KPS) and keystroke 
    data.

    \vspace{10pt}\hspace{20pt}Initiating and stopping both Arduinos will be controlled by 
    the computer-side program. To begin operation, the key H must be pressed. When this 
    key is pressed, the computer-side program will send the solenoid Arduino a signal 
    (0xFFFF), which will then be forwarded to the display Arduino via the TX and RX pins. 
    Termination will follow a similar procedure.
    \hwline
\end{homeworkProblem}

\pagebreak

%%%%%%%%%%%%%%%%%%%%%%%%%%%%%%%%%%%%%%%%%%%%%%%%%%%%%%%%%%%%%%%%%%%%%%%%%%%%%%%%%%%%%%%%%

\begin{homeworkProblem}{}
    \subsection{4 Expected Inputs/Outputs}

    \textbf{Expected Inputs}
    \begin{enumerate}
        \item \vspace{-8pt}\textbf{Keyboard Input:}
        The user would input commands through the keyboard. For example, pressing the key 
        H to start the system and the key J to terminate it.
        \item \vspace{-8pt}\textbf{Beatmap Hit Data (.osu file):}
        This data would be inputted into the computer-side program, to provide the timing
        and sequence of hits in the osu beatmap.
    \end{enumerate}

    \textbf{Expected Outputs}
    \begin{enumerate}
        \item \vspace{-8pt}\textbf{Solenoid Activation:}
        The solenoid Arduino would output signals to activate the solenoids corresponding
        to the hit data received from the computer-side program. This would physically
        press keys on the keyboard.
        \item \vspace{-8pt}\textbf{Display Data:}
        The display Arduino would output data to the connected display, showing
        information such as keys per second and recent key presses. This provides visual
        feedback to the user. KPS on a 16x2 LCD, key strokes on a Nokia 5110 LCD.
        \item \vspace{-8pt}\textbf{Status Indicators:}
        The computer-side program may output status indicators or messages to inform the
        user about the system's operation, such as when it has started or terminated
        successfully.
    \end{enumerate}
    \hwline
\end{homeworkProblem}

%%%%%%%%%%%%%%%%%%%%%%%%%%%%%%%%%%%%%%%%%%%%%%%%%%%%%%%%%%%%%%%%%%%%%%%%%%%%%%%%%%%%%%%%%

\begin{homeworkProblem}{}
    \vspace{-20pt}\subsection{5 Original Work Justification}
    \hspace{20pt}This project presents an original Arduino/C++ endeavor through its 
    fusion of hardware 
    and software elements to enhance the Osu Mania gaming experience. By integrating 
    custom 
    hardware components like Arduino microcontrollers and solenoid arrays, the system 
    goes beyond typical software-based modifications, allowing for physical 
    interaction with the game. It processes real-time data, including beatmap 
    hit timings and key presses, to synchronize physical actions with gameplay, 
    showcasing advanced programming techniques in C++ (multi-threading, chrono timing, 
    windows serial handles). Managing communication and synchronization between multiple 
    components, including the computer-side program, Arduinos, solenoid array, and 
    displays, demonstrates a sophisticated understanding of system architecture and 
    programming principles. Overall, this project stands out as an original and 
    innovative Arduino/C++ project, offering a creative VSRG gaming experience.
    \hwline
\end{homeworkProblem}

\pagebreak

%%%%%%%%%%%%%%%%%%%%%%%%%%%%%%%%%%%%%%%%%%%%%%%%%%%%%%%%%%%%%%%%%%%%%%%%%%%%%%%%%%%%%%%%%

\begin{homeworkProblem}{}
    \subsection{6 Timeline of Development}
    \begin{enumerate}
        \item \vspace{-8pt}\textbf{Week 9 (03/11/24):}\\
        Ryan will be focused on finishing up the display Arduino code,
        and will be finishing up the computer-side program for the display
        Arduino.\\
        John's circuit catches on fire after connecting a 950 mA solenoid with a 600 mA
        transistor.
        \item \vspace{-8pt}\textbf{Week 10 (03/18/24):}\\
        Ryan is helping John with the solenoid computer-side code. \\
        John is researching and prototyping solenoid Arduino.
        \item \vspace{-8pt}\textbf{Week 11 (03/25/24):}\\
        Ryan will try making a GUI, otherwise will help John. \\
        John will try finishing up the software side for the solenoid.
        \item \vspace{-8pt}\textbf{Week 12 (04/01/24):}\\
        Ryan will try making a GUI, otherwise will help John. \\
        John will try finishing up the software side for the solenoid.
        \item \vspace{-8pt}\textbf{Week 13 (04/08/24):}\\
        Ryan will try making a GUI, otherwise will help John. \\
        John is finishing up solenoid Arduino. \\
        Make final stand for both Arduinos to sit on.
        \item \vspace{-8pt}\textbf{Week 14 (04/15/24):}\\
        Create presentation slides and present on 04/15/24 in lab.
        \item \vspace{-8pt}\textbf{Week 15 (04/22/24):}\\
        Do finishing tweaks and do project demonstration on 04/22/24 in
        lab.
        \item \vspace{-8pt}\textbf{Week 16 (04/29/24):}\\
        Do individual group reflection, and submit final report.
    \end{enumerate}
    \hwline
\end{homeworkProblem}

\pagebreak

%%%%%%%%%%%%%%%%%%%%%%%%%%%%%%%%%%%%%%%%%%%%%%%%%%%%%%%%%%%%%%%%%%%%%%%%%%%%%%%%%%%%%%%%%

\begin{homeworkProblem}{}
    \subsection{7 List of Materials Expected}
    \textbf{Display Arduino}
    \begin{enumerate}
        \item \vspace{-8pt} 1x Arduino UNO-R3
        \item \vspace{-8pt} 1x 16x2 LCD
        \item \vspace{-8pt} 1x Nokia 5110 LCD
        \item \vspace{-8pt} 1x 10K$\Omega$ potentiometer
        \item \vspace{-8pt} 4x 10K$\Omega$ resistors
        \item \vspace{-8pt} 1x 220$\Omega$ resistor
        \item \vspace{-8pt} 1x 330$\Omega$ resistor
        \item \vspace{-8pt} 1x 1K$\Omega$ resistor
    \end{enumerate}
    \textbf{Solenoid Array Arduino}
    \begin{enumerate}
        \item \vspace{-8pt} 1x Arduino UNO-R3
        \item \vspace{-8pt} 1x 1K$\Omega$ resistor
        \item \vspace{-8pt} 4x 1n4007 Diode Rectifier
        \item \vspace{-8pt} 4x NPN TIP122 Transistors
        \item \vspace{-8pt} 4x BM-0530B Solenoid
        \item \vspace{-8pt} 12V Barrel-Jack Power Supply
    \end{enumerate}
    \textbf{Miscellaneous}
    \begin{enumerate}
        \item \vspace{-8pt} C++23 Compiler
        \item \vspace{-8pt} Windows Computer
        \item \vspace{-8pt} Lots of jumper wires.
    \end{enumerate}
    \hwline
\end{homeworkProblem}

%%%%%%%%%%%%%%%%%%%%%%%%%%%%%%%%%%%%%%%%%%%%%%%%%%%%%%%%%%%%%%%%%%%%%%%%%%%%%%%%%%%%%%%%%

\begin{homeworkProblem}{}
    \vspace{-20pt}\subsection{8 List of References}
    \begin{enumerate}
        \item \vspace{-8pt}
        \href{https://en.cppreference.com/w/cpp/header/chrono}
        {https://en.cppreference.com/w/cpp/header/chrono}
        \item \vspace{-8pt}
        \href{https://en.cppreference.com/w/cpp/header/thread}
        {https://en.cppreference.com/w/cpp/header/thread}
        \item \vspace{-8pt}
        \href{https://learn.microsoft.com/en-us/windows/win32/api/winbase/ns-winbase-dcb}
        {https://learn.microsoft.com/en-us/windows/win32/api/winbase/ns-winbase-dcb}
        \item \vspace{-8pt}
        \href{https://www.theengineeringprojects.com/2019/06/introduction-to-pn2222.html}
        {https://www.theengineeringprojects.com/2019/06/introduction-to-pn2222.html}
        \item \vspace{-8pt}
        \href{https://www.youtube.com/watch?v=haUWO7BGYTE}
        {https://www.youtube.com/watch?v=haUWO7BGYTE}
        \item \vspace{-8pt}
        \href{https://www.build-electronic-circuits.com/rectifier-diode/}
        {https://www.build-electronic-circuits.com/rectifier-diode/}
        \item \vspace{-8pt}
        \href{https://users.ece.utexas.edu/~valvano/Datasheets/PN2222-D.pdf}
        {https://users.ece.utexas.edu/~valvano/Datasheets/PN2222-D.pdf}
        \item \vspace{-8pt}
        \href{https://www.vishay.com/docs/88503/1n4001.pdf}
        {https://www.vishay.com/docs/88503/1n4001.pdf}
        \item \vspace{-8pt}
        \href{https://www.onsemi.com/pdf/datasheet/tip120-d.pdf}
        {https://www.onsemi.com/pdf/datasheet/tip120-d.pdf}
    \end{enumerate}
    \hwline
\end{homeworkProblem}

\pagebreak

%%%%%%%%%%%%%%%%%%%%%%%%%%%%%%%%%%%%%%%%%%%%%%%%%%%%%%%%%%%%%%%%%%%%%%%%%%%%%%%%%%%%%%%%%

\begin{homeworkProblem}{}
    \subsection{9 Diagrams + Pseudocode}

    Keystrokes and KPS Arduino:
    \img{0.5}{ryan.png}
    \hwline
    Solenoid Array Arduino:
    \img{0.6}{john.png}
    \hwline

    \pagebreak

    Keystrokes and KPS Arduino:
    \begin{enumerate}
        \item \vspace{-5pt}
        Setup pins for both displays.
        \item \vspace{-5pt}
        Wait for start signal from RX pin to start.
        \item \vspace{-5pt}
        Loop until terminate signal read from RX.
        \begin{enumerate}
            \item \vspace{-5pt}
            Check if serial has data, read a short from it.
            \item \vspace{-5pt}
            Using bitwise operators, get the KPS from the first 8 bits,
            and get the key presses from the last 8 bits.
            \item \vspace{-5pt}
            Update displays with received data.
        \end{enumerate}
    \end{enumerate}
    \hwline
    Solenoid Array Arduino:
    \begin{enumerate}
        \item \vspace{-5pt}
        Setup pins for solenoids.
        \item \vspace{-5pt}
        Wait for start signal from RX pin to start.
        \item \vspace{-5pt}
        Loop until terminate signal read from RX.
        \begin{enumerate}
            \item \vspace{-5pt}
            Read in hit press data when available.
            \item \vspace{-5pt}
            Send correct signals to solenoids depending on next hit object
            in queue.
        \end{enumerate}
    \end{enumerate}
    \hwline
\end{homeworkProblem}

%%%%%%%%%%%%%%%%%%%%%%%%%%%%%%%%%%%%%%%%%%%%%%%%%%%%%%%%%%%%%%%%%%%%%%%%%%%%%%%%%%%%%%%%%

\end{document}