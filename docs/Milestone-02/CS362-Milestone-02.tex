\documentclass[11pt]{article}
\usepackage{minted}
\usepackage{amsfonts, amssymb, amsmath, float}
\usepackage{enumerate, esint, nicefrac, algorithm2e}
\parindent 0px
\date{\today}
\title{CS 362 :\hspace{2px}: Milestone 2}
\author{Ryan Magdaleno \& John Ezra See\\ More info on page 2.}

% Helpful ::
% \line(1,0){358px}

\begin{document}
\maketitle
\hspace{20px}Our group would like to create an Arduino system that displays certain 
stats about key presses and also a solenoid key presser. The solenoid machine would 
press keys on the keyboard and act like an automatic rhythm game player. The KPS 
stat display would then display the key presses per second and the animator display 
would show the keystrokes animating upwards. This project is centered around rhythm
games  and the solenoid machine would press to the notes of some rhythm game, with
the stat displays calculating the keys per second and keystrokes based on the 
keyboard presses.

\hspace{20px}The first Arduino is responsible for the displays (corresponding to 
the first and second ideas below). The second Arduino will be responsible for the 
solenoid machine. The solenoid machine consists of a sensor to signify when to press 
the keys (by pixel/light changes) which also communicates with the other 
Arduino dedicated to the displays. We are planning to use a 2x16 LCD to display the
key/second and a Nokia 5110 LCD to display the actual keystrokes. Both of the 
displays will take input from the computer, which is taking input from the 
key-pressing machine. The communication mechanism will be via the serial cable, 
from the solenoid Arduino machine into the computer side and finally into the 
stat display Arduino. The input devices would be the keyboard and photoresistors.
The output device would be the solenoids, 2x16 LCD, and the Nokia 5110 LCD.

\hspace{20px}This project is original because it's centered around the game Osu! mania.
We are making use of a solenoid machine to press the keys for us and display those
stats to whoever is viewing them. This project makes use of the Osu Mania file
format (.osu), which we assume is an original idea considering Osu Mania is a fairly
niche rhythm game. The 2D keystrokes animator display is also original because
we've never seen something like that made on an Arduino display before.

\pagebreak

\textbf{Furthur Information : :} \\
Grout Project Title : : VSRG-UNO-R3 : :  VSRG Stats Display for Arduino. \\
\vspace{10px}\line(1,0){358px} \\Group members : : \\
Ryan Magdaleno (rmagd2) - 11 PM Lab \\
John Ezra See (jsee4) - 1 PM Lab \\
\vspace{10px}\line(1,0){358px} \\First Arduino : : \\
Display : : Key Presses/second display. (2x16 LCD) \\
Display : : Keystrokes/key display animator. (Nokia 5110 LCD) \\
\vspace{10px}\line(1,0){358px} \\Second Arduino : :
Key-press machine - A sensor that detects pixel changes and then communicates 
with a machine (solenoids and photoresistors). \\
\vspace{10px}\line(1,0){358px} \\Group ideas : : \\
First idea - Display : :  Key Presses/second display. \\
Second idea -  Display : : Keystrokes/key display animator. \\
Third idea - Solenoid machine. (detects pixel changes then communicates with
machine) \\
\vspace{10px}\line(1,0){358px} \\External devices : :
\begin{enumerate}
\item Photoresistors.
\item Solenoids.
\item 2x16 LCD.
\item Nokia 5110 LCD.
\end{enumerate}
\end{document}